\section{Auswertung}
\label{sec:Auswertung}
%Siehe \autoref{fig:plot}!

\subsection{Die Modenkurve}
\label{sec:ausw_moden}
In diesem Teil werden die Moden mithilfe eines Oszilloskops untersucht.
Die aufgenommenen Werte zu den Moden sind in \autoref{tab:moden_messwerte} gelistet.
%tabelle messwerte
\begin{table}
    \centering
    \caption{Messwerte zu den drei Moden, wobei der erste Wert das Maximum beschreibt und die anderen beiden den Punkt links bzw. rechts vom Maximum bei dem die Spannung die Hälfte des Maximums beträgt.}
    \begin{tabular}{c c c}
        \toprule
        Modennr. & $U \,/\, V$ & $A \,/\, V$ \\
        \midrule
        1 & $220$ & $31.25$ \\
        & $205$ & $0$ \\
        & $240$ & $0$ \\
        \hline
        2 & $140$ & $21.25$ \\
        & $120$ & $0$ \\
        & $150$ & $0$ \\
        \hline
        3 & $85$ & $17$ \\
        & $70$ & $0$ \\
        & $95$ & $0$ \\
        \bottomrule
    \end{tabular}
    \label{tab:moden_messwerte}
\end{table}
\FloatBarrier
Es liegt ein vollständig bestimmtes LGS mit drei Punkten je Mode vor.
Eine Mode wird durch eine Parabel
\begin{equation*}
    y_i = x \cdot U_i^2 + y \cdot U_i + z
\end{equation*}
beschrieben, wobei $x, y, z$ die zu bestimmenden Parameter und $y_i, U_i$ die gemessenen Datenpunkte sind.
Das LGS ist trivial zu lösen und wird hier nicht weiter ausgeführt.
Die Lösung kann in \autoref{fig:moden} graphisch und in \autoref{tab:moden_ergebnisse} quantitativ betrachtet werden.
\begin{figure}
    \centering
    \includegraphics[width=0.8\textwidth]{content/data/moden.pdf}
    \caption{Die Messwerte $x$ mit zugehöriger Fit-Kurve (Ausgleichsparabel) für die drei Moden. \cite{matplotlib}\cite{numpy}}
    \label{fig:moden}
\end{figure}

%tabelle ergebnisse
\begin{table}
    \centering
    \caption{Die Parameter $x, y, z$ beschreiben die Ausgleichsparabel einer Mode.}
    \begin{tabular}{c c c c}
        \toprule
        Modennr. & $x$ & $y$ & $z$ \\
        \midrule
        1 & $-0.104$ & $46.354$ & $-5125.000$ \\
        2 & $-0.106$ & $28.688$ & $-1912.500$ \\
        3 & $-0.113$ & $18.700$ & $-753.667$ \\
        \bottomrule
    \end{tabular}
    \label{tab:moden_ergebnisse}
\end{table}
\FloatBarrier

\subsection{Die elektronische Bandbreite und Abstimm-Empfindlichkeit}
Die Daten zur elektronischen Abstimmung für den höchsten Modus bei $\sim \SI{9000}{\mega\hertz}$ sind in \autoref{tab:elektronische_abstimmung} aufgeführt.
\begin{table}
    \centering
    \caption{Daten zur elektronischen Abstimmung mit Reflektorspannung $U$ und Frequenz $f$, wobei $a)$ den Peak, $b)$ bzw. $c)$ die Hälfte des Peaks rechts bzw. links vom Maximum beschreibt.}
    \begin{tabular}{c c c c}
        \toprule
        & a) & b) & c) \\
        \midrule
        $U \,/\, \si{\volt}$ & $220$ & $210$ & $230$ \\
        $f \,/\, \si{\mega\hertz}$ & $9010$ & $8989$ & $9029$ \\
    \end{tabular}
    \label{tab:elektronische_abstimmung}
\end{table}
Die elektronische Bandbreite (siehe \autoref{sec:moden_durchfuehrung}) beträgt
\begin{equation*}
    f' - f'' = \SI{40.0}{\mega\hertz}
\end{equation*}
und die Abstimm-Empfindlichkeit liegt bei
\begin{equation*}
    \frac{f' - f''}{U' - U''} = \SI{2.0}{\frac{\mega\hertz}{\volt}} \, .
\end{equation*}
\FloatBarrier

\subsection{Wellenlänge und Phasengeschwindigkeit im Hohlleiter}
\label{sec:ausw_frequenz}
Im Folgenden wird die Wellenlänge und Phasengeschwindigkeit der Mikrowelle bestimmt.
Die Wellenlänge $\lambda_g$ ergibt sich aus dem doppelten Abstand zweier Minima bzw. Maxima und beträgt hier
\begin{align*}
    \lambda_g &= (73 - 48) \cdot 2 \, \si{\milli \metre} \\
    &= \SI{50}{\milli \metre} \, .
\end{align*}
Die Messung wurde bei der Frequenz $f = \SI{9007}{\mega\hertz}$ durchgefürt.
Aus der Frequenz und Wellenlänge der Mikrowelle im Hohlleiter folgt die Phasengeschwindigkeit
\begin{align*}
    \nu_\text{Phase} &= f \cdot \lambda_g \\
    &= \SI{450350000}{\frac{\metre}{\second}} \\
    &= 1,5 \, \mathrm{c} \\
\end{align*}
mit der Lichtgeschwindigkeit $c = \SI{299 792 458}{\metre \per \second}$.
\FloatBarrier

\subsection{Die Dämpfungskurve}
In diesem Teil soll die Dämpfung in Abhängigkeit der Einstellung der Mikrometerschraube untersucht werden.
Die vom SWR-Meter aufgenommene Dämpfung $P$ und Mikrometereinstellung ist in \autoref{tab:daempfung} aufgelistet und in \autoref{fig:daempfung} graphisch dargestellt.
Die Theorie-Kurve wird aus $10$ Punkten der Eichkurve (siehe \autoref{tab:daempfung}) mithilfe einer Ausgleichsparabel (Scipy\cite{scipy}) der Form
\begin{equation*}
    P_\text{eich} = a \cdot x^2 + b \cdot x + c 
\end{equation*}
bestimmt:
\begin{align*}
    a &= \SI{1.59(6)}{} \\
    b &= \SI{0.65(15)}{} \\
    c &= \SI{0.03(8)}{}
\end{align*}

Wie in der Abbildung zu erkennen sind die Messwerte im Vergleich zur Eichkurve stark verschoben.
Es ist ein relativ konstanter Abstand zu erkennen.
Daher wird die Differenz zwischen den Messwerten und der Eichkurve berechnet und der Mittelwert gebildet
\begin{equation}
    \bar{x}_\text{Verschiebung} = \SI{14.38}{\milli\metre}
\end{equation}
und die Messwerte um diesen Wert verschoben.
\begin{table}
    \centering
    \caption{Die gemessene Dämpfung $P$ und die theoretische Dämpfung $P_\text{eich}$ als Funktion der Mikrometerablesung $x$.}
    \label{tab:daempfung}
    \begin{tabular}{S[table-format=1] S[table-format=1.2] S[table-format=1.2]}
        \toprule
        $P \,/\, \si{\dB}$ & $x \,/\, \si{\milli\metre}$ & $P_\text{eich} \,/\, \si{\dB}$\\
        \midrule
        0 & 2.8 & 0 \\
        2 & 3.05 & 0.9 \\
        4 & 3.21 & 1.4 \\
        6 & 3.4 & 1.75 \\
        8 & 3.5 & 2.05 \\
        10 & 3.7 & 2.3 \\
        \bottomrule
    \end{tabular}
\end{table}

\begin{figure}
    \centering
    \includegraphics[width=0.8\textwidth]{content/data/daempfung.pdf}
    \caption{Die theoretische Dämpfungskurve (rot), Messwerte (blau) und die angepassten Messwerte (grün) in Abhängigkeit der Mikrometereinstellung.}
    \label{fig:daempfung}
\end{figure}
\FloatBarrier

\subsection{Das Stehwellenverhältnis}
\label{sec:aus_stehwellen}
Im letzten Teil wird das Stehwellenverhältnis (SWR), also das Verhältnis zwischen maximaler und minimaler Feldstärke im Hohlleiter bestimmt.
Dazu wird ein in der Tiefe verstellbarer Stift in den Hohlleiter geführt, für weitere Erläuterungen oder Erklärungen siehe \autoref{sec:swr}.
Das SWR wird über drei Methoden
\begin{itemize}
    \item direkte Methode
    \item $\SI{3}{\dB}$ -Methode
    \item Abschwächer-Methode
\end{itemize}
bestimmt und später in der Diskussion evaluiert.

\subsubsection{Direkte Methode}
Bei der direkten Methode wird das Stehwellenverhältnis direkt vom SWR-Meter abgelesen.
In \autoref{tab:direkte_methode} ist das SWR in Abhängigkeit der Sondentiefe (am Gleitschraubentransformator) aufgeführt.

\begin{table}
    \centering
    \caption{Das Stehwellenverhältnis (SWR) in Abhängigkeit der Sondentiefe, bestimmt über die direkte Methode.}
    \label{tab:direkte_methode}
    \begin{tabular}{c c}
        \toprule
        Sondentiefe $\,/\, \si{\milli\metre}$ & SWR\\
        \midrule
        $3$ & $1.1$ \\
        $5$ & $1.5$ \\
        $7$ & $3.2$ \\
        $9$ & $\infty$ \\
        \bottomrule
    \end{tabular}
\end{table}
\FloatBarrier

\subsubsection{3 dB-Methode}
Für die $\SI{3}{\dB}$-Methode wird die Sonde $\SI{9}{\milli\metre}$ in den Hohlleiter gefahren.
Die Position des Schlittens bei der das SWR-Meter $\SI{3}{\dB}$ anzeigt, links vom Minimum $d_1$ bzw. rechts vom Minimum $d_2$ werden bei montiertem Abschluss gemessen.
Für den Kurzschluss wird noch die Position zweier benachbarter Minima notiert.
Die Messwerte sind in \autoref{tab:3db_methode} angegeben.

\begin{table}
    \centering
    \caption{Messwerte zur $\SI{3}{\dB}$-Methode.}
    \label{tab:3db_methode}
    \begin{tabular}{c c c c}
        \toprule
        $d_1 \,/\, \si{\milli\metre}$ & $d_2 \,/\, \si{\milli\metre}$ & 1. Min $\,/\, \si{\milli\metre}$ & 2. Min $\,/\, \si{\milli\metre}$ \\
        \midrule
        $63.3$ & $65.3$ & $72.9$ & $48.8$ \\
        \bottomrule
    \end{tabular}
\end{table}
\FloatBarrier

Die Wellenlänge im Hohlleiter lässt sich aus dem doppelten Abstand der Minima bestimmen:
\begin{equation*}
    \lambda_g = (72,9 - 48,8) \cdot 2 \, \si{\milli\metre} = \SI{48.2}{\milli\metre}
\end{equation*}
Daraus folgt nach \autoref{eq:3db_SWR} das Stehwellenverhältnis
\begin{equation}
    S = \SI{7.67}{} \, .
\end{equation}

\subsubsection{Abschwächer-Methode}
Zuletzt wird das SWR für die gleiche Sondentiefe von $\SI{9}{\milli\metre}$ über die Abschwächer-Methode bestimmt.
Das SWR ergibt sich aus der gemessenen bzw. eingestellten Dämpfung im Minimum und Maximum (siehe \autoref{sec:abschwaecher_methode})
\begin{align*}
    A_1 &= \SI{20}{\dB} \\
    A_2 &= \SI{45}{\dB}
\end{align*}
über \autoref{eq:daempfung_SWR} zu
\begin{equation}
    S = \SI{17.78}{} \, .
\end{equation}