\section{Diskussion}
\label{sec:Diskussion}
Auffällig bei der Untersuchung der Moden (\autoref{sec:ausw_moden}) ist, dass die gemessenen Werte im Maximum der Mode nicht im Maximum der Parabel liegen.
Es ist kein systematischer Fehler zu erkennen und es wird von einer relativen Streuung um den Höhepunkt ausgegangen, die durch Messunsicherheiten verursacht wird.
Die Moden wiesen einen gewissen Grad an Asymmetrie auf, was die Messung beeinträchtigt hat.
\\
\\
Die in \autoref{sec:ausw_frequenz} aus der Wellenlänge berechnete Frequenz $f = \SI{8885}{\mega\hertz}$ weicht um $\SI{1.34}{\percent}$ von der eingestellten Frequenz ($\SI{9007}{\mega\hertz}$) ab.
\\
\\
Die Dämpfungskurve (siehe \autoref{fig:daempfung}) beschreibt einen parabelförmigen Verlauf.
Die Messwerte weichen um einen konstanten Wert $\bar{x} = \SI{14.38}{\milli\metre}$ zur Eichkurve ab.
Es ist davon auszugehen, dass die Nullinie des SWR-Meters verschoben ist.
Die relativen Abweichungen der verschobenen Messwerte zur theoretischen Dämpfungskurve ist der Messunsicherheit des empfindlichen SWR-Meters geschuldet.
Zudem kann es sein, dass die Schraube des Gleitschraubentransformators einen Offset hat.
\\
\\
Das durch die direkte Methode bestimmte Stehwellenverhältnis (\autoref{sec:aus_stehwellen}) steigt wie zu erwarten mit der Sondentiefe an.
Für große Welligkeiten ist diese Methode nicht geeignet, was der Wert $S = \infty$ für die maximale Stifttiefe $x = \SI{9}{\milli\metre}$ impliziert.
Zuletzt werden die drei Methoden zur Bestimmung des Stehwellenverhältnis verglichen.
Bei einer Sondentiefe von $\SI{9}{\milli\metre}$ wurden folgende SWR berechnet:
\begin{align*}
    S_\text{direkt} &= \infty \\
    S_\text{3 dB} &= \SI{7.67}{} \\
    S_\text{Abschwächer} &= \SI{17.78}{} \\
\end{align*}
Die drei Methoden weichen stark voneinander ab.
Allgemein kann gesagt werden, dass ein großes Stehwellenverhältnis bei der maximalen Sondentiefe vorliegt.