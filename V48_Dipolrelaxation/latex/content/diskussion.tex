\section{Diskussion}
\label{sec:Diskussion}
Die Aktivierungsenergie für strontiertes Kaliumbromid wird in der Literatur mit $W_\text{lit} = \SI{0.66}{\electronvolt}$\cite{muccillo} angegeben.
Die Abweichungen der experimentell bestimmten Werte zum Theoriewert werden nach
\begin{equation*}
    \Delta X = \frac{X_\text{exp} - X_\text{th}}{X_\text{th}}
\end{equation*}
berechnet und sind in \autoref{tab:abweichungen_w} für die Aktivierungsenergien gelistet.

\begin{table}
    \centering
    \begin{tabular}{c S[table-format=1.1] | S[table-format=1.3] S[table-format=2.2]}
        \toprule
        & $b \,/\, \si{\frac{\kelvin}{\minute}}$ & $W_\text{exp} \,/\, \si{\electronvolt}$ & $\Delta W \,/\, \si{\percent}$ \\
        \midrule
        Methode 1 & 1.5 & 0.79 & 19.7 \\
        & 2.0 & 0.60 & 9.1 \\
        Methode 2 & 1.5 & 0.729 & 10.5 \\
        & 2.0 & 0.884 & 33.9 \\
        \bottomrule
    \end{tabular}
    \caption{Experimentell bestimmte Aktivierungsenergie und Theoriewert im Vergleich.}
    \label{tab:abweichungen_w}
\end{table}
\FloatBarrier

Betrachtet man den simplen Aufbau des Experiments bzw. mit welchen mitteln die Werte bestimmt wurden, so kann man die Abweichungen als akzeptabel einstufen.
\\
\\
Die charakteristische Relaxationszeit wird in der Literatur mit $\tau_{0, \text{th}} = \SI{4e-14}{\second}$\cite{muccillo} aufgeführt.
Die Abweichungen von Experimental- und Theoriewert betragen hier bis zu $\SI{e05}{\percent}$ und es ist eine Streuung in positiver und negativer Richtung zu erkennen.
\\
Ein Grund könnte die Verwendung von fehlerbehafteten Größen sein, die bei Rechnungen durch die Gaußsche Fehlerfortpflanzung beschrieben werden.
Werden nun wiederholt berechnete fehlerbehaftete Größen zur Berechnung neuer Größen verwendet, so kann der Fehler großes Ausmaß annehmen.
Dabei könnte es sich um unbekannte Fehler in den Messwerten handeln, da statistische bereits beachtet wurden.
\\
Wie bereits erwähnt, trägt auch der simple Aufbau des Experiments und das Ablesen der Werte nach Augenmaß zu den Fehlern bei. 
Zum Beispiel wurde die Heizrate manuell über den Heizstrom geregelt und es konnte keine konstante Steigung der Temperatur (gerade im Anfangsbereich) erreicht werden.
\\
Um genauere Ergebnisse zu erzielen, könnte der Heizstrom automatisch an die Temperatur angepasst werden.
Zudem sollte die Temperatur in kleineren Intervallen gemessen werden, um gerade an den Peaks mehr Messpunkte zu erhalten, auch dies könnte automatisiert werden.