\section{Diskussion}
\label{sec:Diskussion}
\begin{table}
    \centering
    \caption{Die berechneten Landé-Faktoren, sowie die korrespondierenden Theorie Werte. Sowie die Art der Polisierung und die Wellenlänge des genutzten Lichts und die Stärke des genutzten Magnetfeldes. Zum Vergleich sind auch die optimalen Magnetfeldstärken angegeben.}
    \begin{tabular}{ccccccc}
        \toprule
        $\lambda / \si{\nano\meter}$ & Polarisierung & $B_\text{exp} \, /\, \si{\milli\tesla}$ &$B_\text{theo} \,/ \,\si{\milli\tesla}$& $\text{Exp.}\,g_\text{ij}$ & $\text{Theo.}\, g _\text{ij}$ & $\text{Abweichung}\,/\,\%$\\
        \midrule
        480.0 & $\pi$    & 443 &  1254.2& $\SI{0.672(25)}{}$  & 0.5   &   34.4\\
        480.0 & $\sigma$ & 365 &  365.0 & $\SI{1.540(33)}{}$  & 1.75  &    12.0\\
        643.8 & $\sigma$ & 443 &  632.2 & $\SI{0.613(25)}{}$  & 1.0   &   38.7\\
        \bottomrule
    \end{tabular}
    \label{tab:Lande-Faktor_disk}
\end{table}
Die aufgenommenen Messwerte und die daraus berechneten Landé-Faktoren sind in \autoref{tab:Lande-Faktor_disk} zu finden.
Dabei ist die Abweichung des Landé-Faktors der blauen $\sigma$-Polarisierten Linie am geringsten.
Dies ist schlüssig, da die berechnete B-Feld Stärke für diese Linie, die Einzige war, die mit dem gegebenen Magneten tatsächlich erreicht werden konnte.
Aus \autoref{tab:Magnetfeld} ist zu entnehmen, dass der Magnet das stärkste mögliche Feld schon bei $B = \SI{443}{\milli\tesla}$ erreicht.
Aufgrund dieser Einschränkung konnten nicht die optimale Aufspaltung aller Spektrallinien erreicht werden.
Wodurch die größeren Abweichung der Spektrallinien blau-$\pi$ und rot-$\sigma$ zu erläutert sind.
Zudem ist die Abschätzung der Maxima von Hand nicht immer eindeutig, da wie in \autoref{fig:pi-blau} und \autoref{fig:sigma-blau} zu erkennen ist, die Maxima oft verwaschen und unscharf sind.
Diese Unklarheit ist zwar in die Berechnung der Landé-Faktoren in Form eines systematischen Fehlers von $\pm 5$ Pixeln eingeflossen.
Allerdings ist davon auszugehen, dass der Fehler bei einigen Messungen auch größer als der angenommene systematische Fehler ist.
\\\\
Für die Spektrallinie blau-$\sigma$ sind zwei Landé-Fakotren $g_1 = 1.5$ und $g_2 = 2$ vorhanden.
In der \autoref{tab:Lande-Faktor_disk} wird dabei der Mittelwert aus $g_1$ und $g_2$ als Theoriewert verwendet.
Die Abweichung zu $g_1$ beträgt $2.6\%$ und zu $g_2$ beträgt sie $23\%$.
