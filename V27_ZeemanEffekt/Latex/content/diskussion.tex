\section{Diskussion}
\label{sec:Diskussion}
\begin{table}
    \caption{Die berechneten Landé-Faktoren, sowie die korrespondierenden Theorie Werte. Sowie die Art der Polisierung und die Wellenlänge des genutzten Lichts und die Stärke des genutzten Magnetfeldes. Zum Vergleich sind auch die optimalen Magnetfeldstärken angegeben.}
    \begin{tabular}{ccccccc}
        \toprule
        $\lambda / \si{\nano\meter}$ & Polarisierung & $B_\text{exp} \, /\, \si{\milli\tesla}$ &$B_\text{theo} \,/ \,\si{\milli\tesla}$& $\text{Exp.}\,g_\text{ij}$ & $\text{Theo.}\, g _\text{ij}$ & $\text{Abw.}\,/\,\%$\\
        \midrule
        480.0 & $\pi$    & 443 &  1254.2& $\SI{0.672(25)}{}$  & 0.5   &   34.4\\
        480.0 & $\sigma$ & 365 &  365.0 & $\SI{1.540(33)}{}$  & 1.75  &   12.0\\
        643.8 & $\sigma$ & 443 &  632.2 & $\SI{1.143(11)}{}$  & 1.0   &   14.3\\
        \bottomrule
    \end{tabular}
    \label{tab:Lande-Faktor_disk}
\end{table}
Die aufgenommenen Messwerte und die daraus berechneten Landé-Faktoren sind in \autoref{tab:Lande-Faktor_disk} zu finden.
Dabei ist die Abweichung des Landé-Faktors der blauen $\sigma$-Polarisierten Linie am geringsten.
Dies ist schlüssig, da die berechnete B-Feld Stärke für diese Linie, die Einzige war, die mit dem gegebenen Magneten tatsächlich erreicht werden konnte.
Aus \autoref{tab:Magnetfeld} ist zu entnehmen, dass der Magnet das stärkste mögliche Feld schon bei $B = \SI{443}{\milli\tesla}$ erreicht.
Aufgrund dieser Einschränkung konnten nicht die optimale Aufspaltung aller Spektrallinien erreicht werden.
Wodurch die größeren Abweichungen der Spektrallinien blau-$\pi$ und rot-$\sigma$ zu erkären sind.
Zudem ist die Abschätzung der Maxima von Hand nicht immer eindeutig, da wie in \autoref{fig:pi-blau} und \autoref{fig:sigma-blau} zu erkennen ist, die Maxima oft "verwaschen" und unscharf sind.
Diese Unklarheit ist zwar in die Berechnung der Landé-Faktoren in Form einer Messunsicherheit von $\pm 5$ Pixeln für die blauen Linien und $\pm 2$ Pixeln für die rote Linie eingeflossen.
Allerdings ist davon auszugehen, dass der Fehler besonders bei den Messungen der blauen Linien auch teilweise größer als der angenommene systematische Fehler ist.
\\\\
Für die Spektrallinie blau-$\sigma$ stehen zwei Landé-Faktoren $g_1 = 1.5$ und $g_2 = 2$ zur Verfügung.
In der \autoref{tab:Lande-Faktor_disk} wird dabei der Mittelwert aus $g_1$ und $g_2$ als Theoriewert verwendet.
Die einzelnen bzw. spezifischen Abweichungen zu $g_1$ beträgt $2.6\%$ und zu $g_2$ beträgt sie $23\%$.
