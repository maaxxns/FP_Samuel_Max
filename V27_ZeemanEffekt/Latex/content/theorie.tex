\section{Theorie}
\label{sec:Theorie}

%Quantenzahlen
\subsection{Gesamtwellenfunktion und Quantenzahlen}
Die Lösung des Wasserstoffproblems liefert die Wellenfunktion
\begin{equation*}
    \Psi_{n, l, m} (x, y, z) = R_{n, l}(r) \cdot Y_l^m (\vartheta, \varphi) \cdot \chi_{m_s}
\end{equation*}
, welche die Verteilung der Aufenthaltswahrscheinlichkeit eines Elektronens angibt.\\
Die Quantenzahl $n = 1, 2, 3 ...$ wird als Hauptquantenzahl bezeichnet, diese gibt die Elektronenschale bzw. das Energieniveau an zu dem der Zustand des Elektrons gehört.
Die Nebenquantenzahl $l = 0, 1, 2, ..., n-1 $ wird auch Drehimpulsquantenzahl genannt, da dieser nach \autoref{eqn:eigenwert_l} den Eigenwert vom Quadrat des Drehimpulsoperators angibt. 
Um die räumliche Orientierung des Drehimpulses anzugeben, wird die Magnetische Quantenzahl $m = \frac{L_z}{\hbar} = -l, - (l-1), ..., (l-1), l$ verwendet.
Die Spinquantenzahl $s = -\frac{1}{2} , \frac{1}{2}$ gibt den Spin an.
\\ \\
Der Bahndrehimpuls und der Elektronenspin stehen wie folgt mit den Quantenzahlen in Relation:
\begin{align}
    |\vec{l}| &= \sqrt{l(l+1)} \hbar \label{eqn:eigenwert_l} \\
    |\vec{s}| &= \sqrt{s(s+1)} \hbar \label{eqn:eigenwert_s}
\end{align}

%Normaler Zeeman
\subsection{Normaler Zeeman-Effekt}
\subsubsection{Allgemein}
Wird ein Atom in ein Magnetfeld $B$ gebracht, so spalten sich die $(2l + 1)$ entarteten Spektrallinien in $(2l + 1)$ äquidistante Energieniveaus auf (siehe Abb. \ref{fig:zeeman_aufspaltung}).
Die Aufspaltung der Spektrallinien, die durch das magnetischen Moment des Bahndrehimpuls $l$ \eqref{eqn:eigenwert_l} erzeugt wird, wird als normaler Zeeman-Effekt bezeichnet.
Das Bohr'sche Magneton
\begin{equation}
    \mu_B = \frac{e \cdot \hbar}{2 m_e}
    \label{eqn:magneton}
\end{equation}
gibt im Bohr'schen Atommodell den Betrag des magnetischen Moments eines Elektrons im Grundzustand an.
Hierbei steht $e$ für die Elementarladung, $\hbar$ für das reduzierte Plancksche Wirkungsquantum und $m_e$ für die Ruhemasse des Elektrons.

Der Abstand der Zeeman-Komponenten ist konstant und beträgt
\begin{equation}
    \Delta E = E_{n,l,m} - E_{n,l,m-1} = \mu_B \cdot B \, .
    \label{eqn:energie_dif_normal}
\end{equation}
\\
\begin{figure}
    \centering
    \includegraphics[width=0.8\textwidth]{content/data/zeeman_aufspaltung.png}
    \caption{Schematische Darstellung des Zeeman-Effekts. Die Spektrallinien spalten sich unter Einfluss eines Magnetfeldes auf.} %ref
    \label{fig:zeeman_aufspaltung}
\end{figure}

Das magnetische Bahnmoment des Elektrons bzw. das magnetische Moment des Drehimpulses
\begin{equation}
    \vec{\mu_l} = -\frac{\mu_B}{\hbar} \cdot \vec{l}
    \label{eqn:magn_moment_l}
\end{equation}
ergibt sich aus dem Bohrschen Magneton $\mu_B$ \eqref{eqn:magneton}, dem reduzierten plankschen Wirkungsquantum $\hbar$ und dem Einheitsvektor $\vec{l}$.
 
\subsubsection{Emission und Absorption von Licht}
Betrachtet man Emission und Absorption von Licht durch Atome in einem Magnetfeld mit $\vec{B} = B \cdot \vec{z}$ wird folgendes beobachtet:

\begin{itemize}
    \item fällt $\sigma^+$-polarisiertes Licht in $z$-Richtung auf ein Atom, so treten Übergänge mit $\Delta m = +1$ auf
    \item analog dazu treten bei $\sigma^-$-polarisiertem Licht Übergänge mit $\Delta m = -1$ auf
    \item bei der Emission von Photonen, werden dem entsprechend $\sigma^+ -$ und $\sigma^-$-Licht in Feldrichtung $\vec{z}$ beobachtet
    \item senkrecht zum Magnetfeld werden drei linear polarisierte Komponenten gemessen (2 senkrecht und eine parallel zu $\vec{z}$)
    \item die Spektrallinien werden bei einem Übergang zwischen zwei Zuständen in drei Zeeman-Komponenten aufgespalten: $\sigma^+$, $\sigma^-$, $\pi$ -Polarisation
\end{itemize}

\subsection{Annomaler Zeeman-Effekt}
Der anomale Zeeman-Effekt tritt auf, wenn der Gesamtspin $\vec{S} = \sum_i \vec{s_i}$ eines Atoms ungleich null ist.
Die Aufspaltung der Spektrallinien ist in diesem Fall komplizierter, da der Elektronenspin mit einem magnetischen Moment
\begin{equation}
    \mu_s = -g_s \cdot \frac{\mu_B}{\hbar} \cdot \vec{s}
    \label{eqn:lande_s}
\end{equation}
(analog zum Drehimpuls) verbunden ist.
Dabei wird $g_s$ als (Spin-)Landé-Faktor bezeichnet, wobei $g_s \approx 2$ aus der Dirac-Theorie folgt.
\\
Der Gesamtdrehimpuls $\vec{j} = \vec{l} + \vec{s}$ ist ohne Magnetfeld zeitlich konstant, also behält Richtung und Betrag bei.
Das gesamte magnetische Moment
\begin{align}
    \vec{\mu_j} &= \vec{\mu_l} + \vec{\mu_s} \\
    &= - \frac{e}{2m_e}(\vec{l} + g_s \vec{s})
\end{align}
präzediert um $\vec{j}$, da $\vec{s}$ im atomaren Magnetfeld (erzeugt durch die Bahnbewegung des Elektrons) präzidiert.
\\
Die Projektion von $\vec{\mu_j}$ auf $\vec{j}$, auch als das mittlere magnetische Moment bezeichnet
\begin{align}
    <\mu_j> &= \frac{\mu_j \cdot \vec{j}}{|\vec{j}|}\\
    &= -\frac{e}{2m_e} \left ( \frac{\vec{l} \cdot \vec{j}}{\vec{|j|}} + g_s \cdot \frac{\vec{s} \cdot \vec{j}}{\vec{|j|}} \right )
\end{align}
wird mithilfe der Produkte $\vec{l} \cdot \vec{j}$ und $\vec{s} \cdot \vec{j}$ wie folgt ausgedrückt:
\begin{align}
    <\mu_j> &= - \frac{3j(j+1) + s(s+1) - l(l+1)}{2 \cdot \sqrt{j(j+1)}} \mu_B \\
    &= - g_j \cdot \sqrt{j(j+1)} \mu_B \\
    &= - g_j \cdot \frac{\mu_B}{\hbar} \cdot \vec{|j|}
\end{align}

Der Landé-Faktor
\begin{equation}
    g_j = 1 + \frac{j(j+1) + s(s+1) - l(l+1)}{2j(j+1)}
    \label{eqn:lande}
\end{equation}
kann als Quotient vom gemessenen magnetischen Moments durch den klassisch theoretisch bestimmten Wert betrachtet werden.
Ist nur ein Bahndrehimpuls vorhanden $\vec{j}=\vec{l}$ bzw. nur ein Spinmoment $\vec{j}=\vec{s}$, so beträgt der Faktor $g_j=1$ bzw. $g_j = g_s \approx 2$.
\\
Im folgenden Teil wird die Annahme getroffen, dass das äußere Magnetfeld $\vec{B}$ schwächer als das durch die Bahnbewegung des Elektrons erzeugte Feld ist.
Dann bleibt die Spin-Bahn-Kopplung ($\vec{j}=\vec{l}+\vec{s}$) erhalten und der Betrag des Gesamtdrehimpulses
\begin{equation}
    |\vec{j}| = \sqrt{j(j+1)} \hbar
\end{equation}
ist im äußeren Magnetfeld konstant.
Die Richtung ändert sich, aufgrund des magnetischen Moments $\vec{\mu_j} = \vec{\mu_l} + \vec{\mu_s}$, welches ein Drehmoment im Magnetfeld erfährt.
Für die z-Komponente des mittleren magnetischen Moments $<\mu_j>$ kann
\begin{equation}
    <\mu_j>_z = - m_j \cdot g_j \cdot \mu_B
\end{equation}
gefolgert werden, wobei $m_j$ halbzahlige Werte im Bereich $-j \leq m_j \leq j$ annehmen kann.
Daraus folgt für die Energie
\begin{align}
    E_{m_j} &= m_j \cdot g_j \cdot \mu_B \cdot \vec{B} + E_0
    \label{eqn:energie_anomal}
\end{align}
, wobei $E_0$ das Energieniveau bei ausgeschaltetem Magnetfeld angibt.
Die Energiedifferenz $\Delta E = E_{m_j} - E{m_{j-1}}$ zwischen zwei Zeeman-Linien kann nach Gleichung \eqref{eqn:energie_anomal} bestimmt werden.
Der Landé-Faktor \eqref{eqn:lande} ist abhängig von den Quantenzahlen $j,l$, daher ist die Aufspaltung der Sepktrallinien für die unterschiedlichen Niveaus verschieden.