\section{Theorie}
\label{sec:Theorie}
\subsection{Gesamtwellenfunktion und Quantenzahlen}
Die Lösung des Wasserstoffproblems liefert die Wellenfunktion
\begin{align*}
    \Psi_{n, l, m} (x, y, z) = R_{n, l}(r) \cdot Y_l^m (\vartheta, \varphi) \cdot \chi_{m_s}
\end{align*}
, welche die Verteilung der Aufenthaltswahrscheinlichkeit eines Elektronens angibt.\\
Die Quantenzahl $n = 1, 2, 3 ...$ wird als Hauptquantenzahl bezeichnet, diese gibt die Elektronenschale an zu dem der Zustand des Elektrons gehört bzw. die Energie.
Die Nebenquantenzahl $l = 0, 1, 2, ..., n-1 $ wird auch Drehimpulsquantenzahl genannt, da dieser nach (Gleichung X) den Eigenwert vom Quadrat des Drehimpulsoperators angibt.%ref
Um die räumliche Orientierung des Drehimpulses anzugeben, wird die Magnetische Quantenzahl $m = \frac{L_z}{\hbar} = -l, - (l-1), ..., (l-1), l$ verwendet.
Die Spinquantenzahl $s = -\frac{1}{2} , \frac{1}{2}$ gibt den Spin an.
\\ \\
Der Bahndrehimpuls und der Elektronenspin stehen wie folgt mit den Quantenzahlen in Relation:
\begin{align}
    |\vec{l}| &= \sqrt{l(l+1)} \hbar \\
    |\vec{s}| &= \sqrt{s(s+1)} \hbar \\
\end{align}

\subsection{Normaler Zeeman-Effekt}
Das Bohr'sche Magneton
\begin{equation}
    \mu_B = \frac{e \cdot \hbar}{2 m_e}
\end{equation}
gibt im Bohr'schen Atommodell den Betrag des magnetischen Moments eines Elektrons im Grundzustand an.
Hierbei steht $e$ für die Elementarladung, $\hbar$ für das Plancksche Wirkungsquantum und $m_e$ für die Masse des Elektrons.