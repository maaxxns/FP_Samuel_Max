\section{Auswertung}
\label{sec:Auswertung}
\subsection{Magnetfeld}
Wie in Abschnitt \ref{sec:Durchführung} beschrieben, wird zunächst die Stärke des Magnetfeldes $B$ in Abhängigkeit des angelegten Stroms $I$ bestimmt.
Die aufgenommenen Messwerte sind in der Tabelle \ref{tab:Magnetfeld} zu sehen.

\begin{table}
    \centering
    \caption{Magnetfeldstärke $B$ in Abhängigkeit des an den Spulen angelegten Stroms $I$}
    \sisetup{table-format=3.1}
    \begin{tabular}{S[table-format=3.1] S S S S}
        \toprule
        $I / \si{\ampere}$ & $B / \si{\milli\tesla}$ &$I / \si{\ampere}$ & $B / \si{\milli\tesla}$ \\
        \midrule
        0   &     7 &   3   & 280 \\
        0.5 &    51 &   3.5 & 325 \\
        1   &    97 &   4   & 367 \\
        1.5 &   142 &   4.5 & 407 \\
        2   &   188 &   5   & 443 \\
        2.5& 235 & & \\
        \bottomrule
    \end{tabular}
    \label{tab:Magnetfeld}
\end{table}

Die aufgenommenen Werte werden in Abbildung \autoref{fig:Magnetfeld} gegeneinander aufgetragen.
Dabei wird die Abbildung mit dem Python Paket 'matplotlib' \cite{matplotlib} erstellt.
Zudem wird eine Ausgleichsrechung für die Messwerte durchgeführt.
Die Rechung wurde mit dem Python Paket 'scipy' \cite{scipy} nach der Funktion
\begin{equation*}
    f(x) = ax^3 +bx^2+cx+d
\end{equation*}
durchgeführt und ergibt für die Koeffizienten:
\begin{align*}
    a =& \SI{-0.766}{\milli\tesla}   \\
    b =& \SI{4.207}{\milli\tesla} \\
    c =& \SI{85.318}{\milli\tesla} \\
    d =& \SI{7.314}{\milli\tesla} \\
\end{align*}

\begin{figure}
    \centering
    \includegraphics[width=\textwidth]{content/data/magnetfeld.pdf}
    \caption{Der angelegte Strom $I \, \text{in} \, \si{\ampere}$ aufgetragen gegen die gemessene Magnetfeldstärke $B \, \text{in} \, \si{\milli\tesla}$.}
    \label{fig:Magnetfeld}
\end{figure}

\subsection{Blaue Spektrallinie}

\subsubsection{\boldmath \texorpdfstring{$\pi$}{pi}-Polarisiertes Licht}
Zunächst wird wie im Abschnitt \ref{sec:Durchführung} beschrieben, die Änderung der blauen Spektrallinie im Magnetfeld aufgenommen.
Das Bild der Aufspaltung für das $\pi -$Polarisierte Licht ist dabei in Abbildung \ref{fig:pi-blau} zu sehen.
In der Grafik wird die Blauen Spektrallinien ohne Einfluss durch ein Magnetfeld und die Spektrallinie des $\pi -$Polarisierten Lichts im Magnetfeld übereinander gelegt.
Oben befindet sich dabei das Licht welches durch das Magnetfeld beeinflusst wird und unten das Licht welches nur durch durch das Prisma gebrochen wird.

\begin{figure}
    \centering
    \includegraphics[width=0.75\textwidth]{content/data/Blau_0_pi_uebernander.JPG}
    \caption{Im oberen Teil des Bilder ist das $\pi -$Polarisierte Licht unter Einfluss des Zeeman-Effekts zu sehen. Im unteren Teil ist die blaue Spektrallinie der Lampe ohne Magnetfeld zum Vergleich zu sehen.}
    \label{fig:pi-blau}
\end{figure}

Zur Messung der Wellenlängenverschiebung werden nun die Abstände zwischen Maxima und Maxima der unbeeinflussten Spektrallinien $\Delta s$
und die Abstände zwischen den Anfängen der Maxima der durch den Zeeman-Effekt beeinflussten Maxima $\delta s$ gemessen.
Die Längen werden dabei in Anzahl von Pixeln gemessen und sind in der Tabelle \ref{tab:blau-pi} zu finden.
Um die Messwerte zu visualisieren werden die Werte in einem Plot in Abbildung \ref{subfig:blau_pi_mess} aufgetragen.
In der linken Grafik sind dabei die Messwerte $\Delta s$ und $\delta s$ zu sehen.
In der Abbildung \autoref{subfig:blau_pi_versch} dagegen die berechnete Wellenlängenverschiebung $\delta \lambda$.

\begin{table}
    \centering
    \caption{Die Abstände der Maxima der Spektrallinien in Anzahl von Pixel.
    $\Delta s$ gibt dabei die unbeeinflussten Abstände und $\delta s$ die durch den Zeeman-Effekt beeinflussten Abstände, des $\pi -$ Polarisirtem Lichts an.}
    \begin{tabular}{cccc}
        \toprule
        Ordnung & $\Delta s \, /$ Pixel  & $\delta s \, /$ Pixel & $\delta \lambda \, / \, \si{\nano\meter}$  \\
        \midrule
        1  &    270 &   56  & $\SI{0.00281(25)}{}$ \\
        2  &    270 &   54  & $\SI{0.00271(25)}{}$ \\
        3  &    224 &   52  & $\SI{0.00315(31)}{}$ \\
        4  &    206 &   54  & $\SI{0.00356(34)}{}$ \\
        5  &    190 &   44  & $\SI{0.00314(36)}{}$ \\
        6  &    182 &   44  & $\SI{0.00328(38)}{}$ \\
        7  &    178 &   44  & $\SI{0.00336(39)}{}$ \\
        8  &    165 &   40  & $\SI{0.00329(42)}{}$ \\
        9  &    150 &   36  & $\SI{0.00326(46)}{}$ \\
        10 &    145 &   36  & $\SI{0.00337(48)}{}$ \\
        11 &    142 &   34  & $\SI{0.00325(49)}{}$ \\
        \bottomrule
    \end{tabular}
    \label{tab:blau-pi}
\end{table}

\begin{figure}
    \caption{Links die Messwerte $\Delta s$ und $\delta s$ gegen die Ordnung geplottet und rechts die berechnete Wellenlaengenverschiebung gegen die Ordnung aufgetragen.}
    \begin{subfigure}{0.48\textwidth}
        \centering
        \includegraphics[height=5cm]{content/data/blau_pi_messwerte.pdf}
        \caption{Messwerte $\Delta s$ und $\delta s$ in Anzahl von Pixeln gegen die Ordnung aufgetragen.}
        \label{subfig:blau_pi_mess}
    \end{subfigure}
    \hfill
    \begin{subfigure}{0.48\textwidth}
        \centering
        \includegraphics[height=5cm]{content/data/blau_pi_verschiebung.pdf}
        \caption{Die berechnete Wellenlängenverschiebung in $\si{\nano\meter}$ in Abhängigkeit von der Ordnung.}
        \label{subfig:blau_pi_versch}
    \end{subfigure}
    \label{fig:blau_pi_mess_versch}
\end{figure}

Um nun die Wellenlängenverschiebung zu berechnen wird die Formel
\begin{equation}
    \delta \lambda = \frac{\delta s \Delta \lambda _\text{D}}{2\Delta s}
   \label{eq:Wellenlaengenverschiebung}
\end{equation}

mit dem Dispersiongebiet

\begin{equation*}
    \Delta \lambda _\text{D} = \frac{\lambda^2}{2d\sqrt{n^2-1}}
\end{equation*}
genutzt. Die Formel wurde aus der Quelle \cite[4]{anleitung} entnommen.
Zudem wird ein systematischer Fehler von $\pm 5$ Pixeln angenommen, da die Maxima der Spektrallinie im Programm \cite{paint3d} abgeschätzt worden sind.
Zur Berechung der Fehler der Wellenlängenverschiebung wird das Python Paket \cite{uncertainties} genutzt.
Dieses nutzt zur Berechung der Fehlerfortpflanzung die Formel
\begin{equation}
    F(\delta \lambda) = \frac{1}{2} \Delta \lambda _\text{D} \sqrt{\left (\frac{1}{\Delta s} F(\delta s) \right)^2 + \left ( \frac{\delta s}{\Delta s^2} F(\delta s) \right )^2}
    \label{eq:fehler_Wellenlängenverschiebung}
\end{equation}
Die berechneteten Werte für die Wellenlängenverschiebung sind in Tabelle \autoref{tab:blau-pi} zu finden.
Es wird über alle Werte der Wellenlängenverschiebung gemittelt daraus ergibt sich der Wert

\begin{align*}
    \delta \lambda _\text{$\pi$-blau} = & \SI{0.00316(011)}{\nano\meter}. \\
\end{align*}

\subsubsection{\boldmath\texorpdfstring{$\sigma$}{sigma} -Polarisiertes Licht}

%Die selben Berechungen wurden für die Messwerte der $\sigma -$Spektrallinien des Blauen Lichts durchgeführt.
Die Aufspaltung der Maxima des Blauen Licht welches $\sigma -$Polarisiert ist, ist in Abbildung \ref{fig:sigma-blau} zu sehen.

\begin{figure}
    \centering
    \includegraphics[width=0.75\textwidth]{content/data/Blue_sigma_0_uebernander.JPG}
    \caption{Im oberen Teil des Bilder ist das $\sigma -$Polarisierte Licht unter Einfluss des Zeeman-Effekts zu sehen. Im unteren Teil ist die blaue Spektrallinie der Lampe ohne Magnetfeld zum Vergleich zu sehen.}
    \label{fig:sigma-blau}
\end{figure}

Zur Bestimmung der Wellenlängenverschiebung wird wie beim $\pi -$Polarisiertem Licht zunächst $\Delta s$ und $\delta s$ gemessen.
Die gemessenen Werte sind in Tabelle \autoref{tab:blau-sigma} zu sehen.
In Abbildung \autoref{subfig:blau_sigma_mess} werden die Messwerte $\Delta s$ und $\delta s$ gegen die Ordnung aufgetragen.
In der Grafik \autoref{subfig:blau_sigma_versch} wird die Wellenlängenverschiebung $\delta \lambda$ in $\si{\nano\meter}$ gegen die Ordnung zu geplottet. 

\begin{table}
    \centering
    \caption{$\Delta s$ der blauen Spektrallinie und $\delta s$ des $\sigma -$Polarisiertem Lichts.}
    \begin{tabular}{cccc}
        \toprule
        Ordnung & $\Delta s \, / $ Pixel & $\delta s \, / $ Pixel & $\delta \lambda \, / \,\si{\nano\meter}$ \\
        \midrule
        1   &   270  &    130   & $\SI{0.00654(27)}{}$   \\
        2   &   264  &    130   & $\SI{0.00669(28)}{}$   \\
        3   &   227  &    109   & $\SI{0.00652(32)}{}$   \\
        4   &   200  &    100   & $\SI{0.00679(79)}{}$   \\
        5   &   176  &    91    & $\SI{0.00702(47)}{}$   \\
        6   &   182  &    91    & $\SI{0.00679(17)}{}$   \\
        7   &   164  &    80    & $\SI{0.00663(61)}{}$   \\
        8   &   161  &    70    & $\SI{0.00591(03)}{}$   \\
        9   &   150  &    55    & $\SI{0.00498(26)}{}$   \\
        10  &   133  &    45    & $\SI{0.00459(39)}{}$   \\
        11  &   130  &    38    & $\SI{0.00397(47)}{}$   \\
        \bottomrule
    \end{tabular}
    \label{tab:blau-sigma}
\end{table}

\begin{figure}
    \caption{Links die Messwerte $\Delta s$ und $\delta s$ gegen die Ordnung geplottet und rechts die berechnete Wellenlaengenverschiebung gegen die Ordnung aufgetragen.}
    \begin{subfigure}{0.48\textwidth}
        \centering
        \includegraphics[height=5cm]{content/data/blau_sigma_messwerte.pdf}
        \caption{Messwerte $\Delta s$ und $\delta s$ in Anzahl von Pixeln gegen die Ordnung aufgetragen.}
        \label{subfig:blau_sigma_mess}
    \end{subfigure}
    \hfill
    \begin{subfigure}{0.48\textwidth}
        \centering
        \includegraphics[height=5cm]{content/data/blau_sigma_verschiebung.pdf}
        \caption{Die berechnete Wellenlängenverschiebung in $\si{\nano\meter}$ in Abhängigkeit von der Ordnung.}
        \label{subfig:blau_sigma_versch}
    \end{subfigure}
    \label{fig:blau_sigma_mess_versch}
\end{figure}

Mit den Werte $\Delta s$ und $\delta s$ wird nun die Wellenlängenverschiebung $\delta \lambda$ berechnet, diese ist ebenfalls in der Tabelle \autoref{tab:blau-sigma} zu sehen.
Zu Berechnung der Wellenlängenverschiebung wird Gleichung \eqref{eq:Wellenlaengenverschiebung} genutzt.
Dabei wird ein systematischer Fehler von $\pm 5$ Pixeln angenommen. 
Hier wird der Fehler nach Gleichung \eqref{eq:fehler_Wellenlaengenverschiebung} bestimmt.
Die Mittlung über alle Wellenlängenverschiebung des $\sigma -$Polarisiertem Lichts ergibt
\begin{align*}
    \delta \lambda _\text{$\sigma$ -blau} =& \SI{0.00621(012)}{\nano\meter}
\end{align*}

\subsection{Rote Spektrallinie}
\subsubsection{\boldmath\texorpdfstring{$\sigma$}{sigma}-Polarisiertes Licht}
Zuletzt wird die Aufspaltung von rotem Licht welches $\sigma -$Polarisiert ist aufgenommen.
Die Aufspaltung ist in der Abildung \autoref{fig:rot-sigma} zu erkennen.

\begin{figure}
    \centering
    \includegraphics[width=0.75\textwidth]{content/data/Rot_0_sigma_uebernander.JPG}
    \caption{Im oberen Teil des Bilder ist das $\sigma -$Polarisierte Licht unter Einfluss des Zeeman-Effekts zu sehen. Im unteren Teil ist die rote Spektrallinie der Lampe ohne Magnetfeld zum Vergleich zu sehen.}
    \label{fig:rot-sigma}
\end{figure}

Es werden die Werte $\Delta s$ und $\delta s$ bestimmt diese sind in Tabelle \autoref{tab:sigma-rot} zu sehen.
Zudem sind die Messwerte garphisch in Abbildung \autoref{subfig:Rot_mess} zu sehen.
In der linken Abbildung \autoref{subfig:Rot_mess} sind die Messwerte $\Delta s$ und $\delta s$ gegen die Ordnung aufgetragen,
in der rechten Abbildung \autoref{subfig:Rot_versch} sind die berechneten Wellenlängenverschiebung zu sehen.
\begin{table}
    \centering
    \caption{$\Delta s$ der roten Spektrallinie und $\delta s$ des $\sigma -$Polarisiertem Lichts.}
    \begin{tabular}{cccc}
        \toprule
        Ordnung & $\Delta s \, /$ Pixel & $\delta s \, /$ Pixel & $\delta \lambda \, / \, \si{\nano\meter}$ \\
        \midrule
        1   &   308  &    76 & $\SI{0.00603(40)}{}$   \\
        2   &   244  &    63 & $\SI{0.00631(51)}{}$   \\
        3   &   228  &    59 & $\SI{0.00632(55)}{}$   \\
        4   &   202  &    59 & $\SI{0.00714(63)}{}$   \\
        5   &   190  &    42 & $\SI{0.00540(65)}{}$   \\
        6   &   177  &    40 & $\SI{0.00552(70)}{}$   \\
        7   &   168  &    34 & $\SI{0.00494(74)}{}$   \\
        8   &   164  &    29 & $\SI{0.00432(75)}{}$   \\
        9   &   147  &    21 & $\SI{0.00349(84)}{}$   \\
        10  &   168  &    21 & $\SI{0.00305(73)}{}$   \\
        \bottomrule
    \end{tabular}
    \label{tab:sigma-rot}
\end{table}

\begin{figure}
    \caption{Links die Messwerte $\Delta s$ und $\delta s$ gegen die Ordnung geplottet und rechts die berechnete Wellenlaengenverschiebung gegen die Ordnung aufgetragen.}
    \begin{subfigure}{0.48\textwidth}
        \centering
        \includegraphics[height=5cm]{content/data/Rot_messwerte.pdf}
        \caption{Messwerte $\Delta s$ und $\delta s$ in Anzahl von Pixeln gegen die Ordnung aufgetragen.}
        \label{subfig:Rot_mess}
    \end{subfigure}
    \hfill
    \begin{subfigure}{0.48\textwidth}
        \centering
        \includegraphics[height=5cm]{content/data/Rot_verschiebung.pdf}
        \caption{Die berechnete Wellenlängenverschiebung in $\si{\nano\meter}$ in Abhängigkeit von der Ordnung.}
        \label{subfig:Rot_versch}
    \end{subfigure}
    \label{fig:Rot_mess_versch}
\end{figure}

Aus den Werte der Tabelle \autoref{tab:sigma-rot} wird nach Gleichung \eqref{eq:Wellenlaengenverschiebung} die Wellenlängenverschiebung berechnet.
Danach wird über alle Werte gemittelt.
Es wird ein systematische Fehler von $\pm 5$ Pixeln angenommen.
Die Wellenlängenverschiebung für die rote $\sigma -$Polarisierte Linie beträgt
\begin{align*}
    \delta \lambda _\text{$\sigma$-rot} =&  \SI{0.00544(020)}{\nano\meter} \\
\end{align*}

Alle gemittelten Wellenlängenverschiebungen sind nocheinmal unten in Tabelle \autoref{tab:Wellenlaengenverschiebung_alle} zusammengefasst zu sehen.

\begin{table}
    \centering
    \caption{Die gemittelten Wellenlängenverschiebung für die blaues $\pi$- und $\sigma$- Polarisiertes, sowie rotes $\sigma$- Polarisiertes Licht unter Einfluss des Zeeman Effekts.}
    \begin{tabular}{cc}
        \midrule
        $\delta \lambda _\text{$\pi$-blau}$ = & $\SI{0.00316(011)}{\nano\meter}$ \\
        $\delta \lambda _\text{$\sigma$ -blau}$ =& $\SI{0.00621(012)}{\nano\meter}$ \\
        $\delta \lambda _\text{$\sigma$-rot}$ =&  $\SI{0.00544(020)}{\nano\meter}$ \\
        \bottomrule
    \end{tabular}
    \label{tab:Wellenlaengenverschiebung_alle}
\end{table}

\subsection{Berechung der Landé-Faktoren}

Zur Berechung der Landé-Faktoren der verschieden Spektrallinie wird die Gleichung \eqref{eq:Lande_Faktor} genutzt.
Es werden die gemittelten Wellenlängenverschiebungen aus Tabelle \autoref{tab:Wellenlaengenverschiebung_alle} verwendet um die Landé-Faktoren zu berechnen.
Der Fehler des Landé-Faktors wird mit 
\begin{equation*}
    F(g_\text{ij}) = \frac{hc}{\lambda^2 \mu _\text{B}} \sqrt{\left ( \frac{1}{B} F(\delta \lambda) \right)^2}
\end{equation*}

Die berechneten Landé-Faktoren sind in Tabelle \autoref{tab:Lande-faktor} zu finden.
Zu den jeweiligen Landé-Faktoren sind dort auch die genutzten $B$-Felder und die Polarisierungen und Wellenlängen beeinflussten Lichts zu sehen.

\begin{table}
    \centering
    \caption{Die berechneten Landé-Faktoren, sowie die Art der Polisierung und die Wellenlänge des genutzten Lichts.}
    \begin{tabular}{cccc}
        \toprule
        $\lambda / \si{\nano\meter}$ & Polarisierung & $B / \si{\milli\tesla}$ & Landé-Faktor \\
        \midrule
        480.0 & $\pi$ & 443 & $\SI{0.672(25)}{}$ \\
        480.0 & $\sigma$ & 365 & $\SI{1.540(33)}{}$ \\
        643.8 & $\sigma$ & 443 & $\SI{0.613(25)}{}$ \\
        \bottomrule
    \end{tabular}
    \label{tab:Lande-Faktor}
\end{table}

