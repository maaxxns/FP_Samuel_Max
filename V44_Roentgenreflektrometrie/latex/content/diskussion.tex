\section{Diskussion}
\label{sec:Diskussion}
Der Detektorscan folgt wie erwartet den Verlauf einer Gauß-Verteilung.
Auch der Z-Scan folgt dem erwarteten Verlauf.
Der Rockingscan sieht zunächst auch gut aus, allerdings weicht der ermitteltete Geometriewinkel stark vom theoretischen Geometriewinkel ab.
Zum Vergleich sind beide Winkel in \autoref{tab:disk_geowinkel} aufgefasst.
\begin{table}
    \centering
    \caption{Der experimentell ermitteltete und theoretische Geometriewinkel.}
    \begin{tabular}{ccc}
        \toprule
        $\alpha_\text{g, exp}$ & $\alpha_\text{g, theo} $ & Abweichung \\
        \midrule
        $\SI{0.560}{\degree}$ & $\SI{0.687}{\degree}$ & $18.48 \, \%$ \\
        \bottomrule
    \end{tabular}
    \label{tab:disk_geowinkel}
\end{table}
Dies kann daran liegen, dass die Justierung nicht perfekt war, oder dass die Proben Angaben aus Quelle \cite{alte_anleitung} nicht mehr aktuelle sind.

Die durch den Parrat-Algorithmus und durch die Minima der Oszillation bestimmte Schichtdicke ist in \autoref{tab:disk_schichtdicke} einzusehen.
\begin{table}
    \centering
    \caption{Die durch den Parratt-Algorithmus und durch die Minima der Oszillation bestimmten Werte für die Schichtdicke.}
    \begin{tabular}{ccc}
        \toprule
        $d_\text{Parratt}$ & $d_\text{Minima} $ & Abweichung \\
        \midrule
        $\SI{86.3}{\nano\meter}$ &$\SI{87.7(023)}{\nano\meter}$& $1.6 \, \%$ \\
        \bottomrule
    \end{tabular}
    \label{tab:disk_schichtdicke}
\end{table}
Es fällt auf, dass die beiden Werte nur sehr gering voneinander abweichen, was die Vermutung Nahe legt, dass die Werte nah an der wahren Schichtdicke liegen.
Die anderen Werte die durch den Parratt-Algorithmus ermittelt wurden sind folgende
\begin{align*}
    \delta_\text{PS, exp} &= 0.5 \cdot 10^{-6}      \\
    \delta_\text{PS, Lit} &= 3.5 \cdot 10^{-6}      \\ 
    \delta_\text{Si, exp} &= 6.2 \cdot 10^{-6}      \\
    \delta_\text{Si, Lit} &= 7.6 \cdot 10^{-6}      \\
    \sigma_\text{Luft, PS, exp} &= \SI{0.85}{\nano\meter} \\ 
    \sigma_\text{PS, Si, exp} &= \SI{0.55}{\nano\meter}\, . \\ 
\end{align*}
Es sind zudem unter dem korrespondierenden $\delta_\text{i}$ Wert einer Schicht die Literaturwerte aufgezeigt.
Für die $\sigma$-Werte konnten leider keine Literaturwerte gefunden werden.
Wie zu sehen ist liegen die experimentell bestimmten Werte in der selben Größenordnungen wie die Literaturwerte.
Die Abweichungen die allerdings bestehen können damit erklärt werden, dass die Kurve des Parratt-Algorithmusses nicht perfekt an die Messwerte angepasst werden konnten.
Da dies ist für einen Menschen nur schwer möglich ist.
Die beiden gemessenen kritsichen Winkel, sowie die korrespondierenden Literaturwerte sind in \autoref{tab:disk_kritwinkel} zu finden.
Zudem ist die Abweichung des experimentell und des Literaturwertes in der Tabelle angegeben.
\begin{table}
    \centering
    \caption{Die bestimmten kristischen Winkel der beiden Stoffe, sowie die Literaturwerte der Winkel. In der letzten Spalte ist zudem die Abweichung zu finden.}
    \begin{tabular}{cccc}
        \toprule
        Stoff & $\alpha_\text{c, exp}$ & $\alpha_\text{c, Lit}$ & Abweichung \\
        \midrule
        Polystyrol  &$\SI{0.057}{\degree}$ &$\SI{0.153}{\degree}$ & $62.7\,\%$ \\
        Sillizium &$\SI{0.201}{\degree}$ & $\SI{0.223}{\degree}$ & $9.9 \, \% $ \\
        \bottomrule
    \end{tabular}
    \label{tab:disk_kritwinkel}
\end{table}
Wie zu sehen ist, weicht der bestimmte kritische Winkel von Polystyrol stark von dem Literaturwert ab.
Der Wert für den kristischen Winkel von Sillizium passt besser mit dem Literaturwerte überein.
Da die Werte aus dem $\delta$-Wert bestimmten wurden ist dies aber nicht weiter verwunderlich, da der $\delta$-Wert von Polystyrol mit $85.7\,\%$ schon stark von seinem Literaturwert abweicht.
Anders als der $\delta$-Wert von Sillizium, welcher nur $18.4\,\%$ von seinem Literaturwert abweicht.
\\\\
Allgemein war es schwierig die Werte des Parratt-Algorithmus perfekt auf die experimentellen Daten anzupassen.
So konnt kein perfekter Fit erstellt werden und es gut möglich, dass mit anderen Parametern ein besserer Fit erzeugt werden kann.
Zudem waren die Amplituden der Kissig-Oszillation sehr gering, was darauf deutet, dass während der Messung ein systematischer Fehler vorhanden war.
Die Ursache des Fehler ist allerdings unbekannt.
